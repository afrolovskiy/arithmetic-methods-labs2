\documentclass[a4paper,12pt]{report}

\usepackage[utf8x]{inputenc}
\usepackage[russian,english]{babel} 
\usepackage[T2A]{fontenc}

\usepackage{mathtext}
\usepackage{tabularx}
\usepackage{array}
\usepackage{amsmath}

%\usepackage{ucs}
\usepackage{geometry}
\geometry{left=2cm}
\geometry{right=1.5cm}
\geometry{top=1cm}
\geometry{bottom=2cm}

\author{Фроловский Алексей \\ ИУ7-17 \\ Вариант 13}
\title{Лабораторная работа №2 \\ по курсу << Mетоды вычислений >>  \\ на тему: \\  
<< Колебания струны >> }
\date{4 июня 2013г.}

% Убираем слово Chapter# перед названием главы
\makeatletter
\def\@makechapterhead#1{%
	\vspace*{50\p@}%
	{
		\parindent \z@ \raggedright \normalfont
		%\ifnum \c@secnumdepth >\m@ne
		%    \huge\bfseries \@chapapp\space \thechapter
		%    \par\nobreak
		%    \vskip 20\p@
		%\fi
		\interlinepenalty\@M
		\Huge \bfseries #1\par\nobreak
		\vskip 40\p@
	}
}
\makeatother

\begin{document}

\maketitle

\chapter{Постановка задачи}
\section{Формулировка задачи}
Найти функцию $u(x, t)$, описывающую поперечные малые колебания однородной струны длины $l = 1$,
концы которой движутся по заданным законам. Значение $u(x, t)$ задает величину отклонения точки
струны с координатой $x$ в момент времени $t$ от положения равновесия. Движение левого конца струны
$(x = 0)$ определяется законом $u(0, t) = \mu(t)$, правого $(x = l)$ - законом $u(l, t) = \nu (t)$. Начальное
положение струны $u(x, 0) = \phi(x)$, начальная скорость $u_{t}(x, 0) = \psi(x)$. Закон колебаний струны
определяется дифференциальным уравнением $u_{tt} = a^{2}u_{xx}$.

\begin{displaymath}
	\left\{
		\begin{array}{l}
			u_{tt}= a^{2}u_{xx}, 0 < x < l, t > 0 \\
			u\vert_{t=0} =  \varphi(x)  \\
			u_{t}\vert_{t=0} =  \psi(x)  \\
			u\vert_{x=0} =  \mu(t) \\
			u\vert_{x=l} =  \nu(t)
		\end{array} \right.
\end{displaymath}

\section{Данные варианта}
Функции, задающие краевые условия задачи, имеют следующий вид:$\varphi(x) = 0,5x(x^{2} + 1)$,
$\psi(x) = cos 2x$, $\mu(t) = 0,5 + 2t - t^{2}$ и $\nu(t) = 1$. Неизвестный параметр $a$ равен $1$.

\chapter{Решение задачи}
\section{Разностная схема}
Данная задача может быть решена с использованием схемы "крест". Тогда разностная схема имеет
следующий вид:

\begin{displaymath}
	\left\{
		\begin{array}{l}
			\dfrac{u_{i}^{j+1} - 2u_{i}^{j} + u_{i}^{j-1}}{\tau^{2}} = a^{2} \dfrac{u_{i-1}^{j} - 2u_{i}^{j} + u_{i+1}^{j}}{h^{2}}, i = \overline{1, N-1}, j= \overline{1, M} \\
			\dfrac{u_{i}^{1} - u_{i}^{0}}{\tau} = \psi_{i} + \dfrac{\tau}{2} \dfrac{u_{i-1}^{0} - 2u_{i}^{0} + u_{i+1}^{0}}{h^{2}}, i= \overline{1, N-1} \\
			u_{i}^{0} = \varphi_{i}, i = \overline{0, N}  \\
			u_{0}^{j} = \mu^{j}, j = \overline{1, M} \\
			u_{N}^{j} = \nu^{j}, j = \overline{1, M}
		\end{array} \right.
\end{displaymath}

Сетка при этом определяется шагами по осям $x$ и $t$:
\begin{equation}
	h = l / N, \tau = T / M, x_{i} = ih, t_{j} = j\tau, i =\overline{0, N}, j = \overline{0, M},
\end{equation}
где $N$ -  количесво шагов по оси $x$, $M$ - количество шагов по оси $t$. $l$ - длина струны, $T$ - время проведений расчётов.

Выбранная разностная схема устойчива, если выполняется следующее неравенство:
\begin{equation}
	\dfrac{a^{2}\tau^{2}}{h^2} \le 1.
\end{equation}

Порядок апроксимации равен $O(\tau^{2} + h^{2})$ в силу замены дифференциальных уравнений
разностными.

Значения функции $u(x, t)$ вычисляются послойно.

\end{document}




