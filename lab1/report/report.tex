\documentclass[a4paper,12pt]{report}

\usepackage[utf8x]{inputenc}
\usepackage[russian,english]{babel} 
\usepackage[T2A]{fontenc}

\usepackage{mathtext}
\usepackage{tabularx}
\usepackage{array}

%\usepackage{ucs}
\usepackage{geometry}
\geometry{left=2cm}
\geometry{right=1.5cm}
\geometry{top=1cm}
\geometry{bottom=2cm}

\author{Фроловский Алексей \\ ИУ7-17 \\ Вариант 13}
\title{Лабораторная работа №1 \\ по курсу << Mетоды вычислений >>  \\ на тему: \\  
<< Распространение тепла в стержне >> }
\date{31 мая 2013г.}

% Убираем слово Chapter# перед названием главы
\makeatletter
\def\@makechapterhead#1{%
	\vspace*{50\p@}%
	{
		\parindent \z@ \raggedright \normalfont
		%\ifnum \c@secnumdepth >\m@ne
		%    \huge\bfseries \@chapapp\space \thechapter
		%    \par\nobreak
		%    \vskip 20\p@
		%\fi
		\interlinepenalty\@M
		\Huge \bfseries #1\par\nobreak
		\vskip 40\p@
	}
}
\makeatother

\begin{document}

\maketitle

\chapter{Постановка задачи}
\section{Формулировка задачи}
Найти температуру $u(x, t)$ тонкого стержня длиной $l$ с теплоизолированной боковой поверхностью, 
на концах которого задан температурный режим: один конец поддерживается при заданной
фиксированной температуре или теплоизолирован, на другой конец извне подается заданный 
тепловой поток. Коэффициент теплопроводности $K$ меняется в зависимости от температуры по 
заданному закону $K = K(u)$. В начальный момент времени $t = 0$ стержень находится при фиксированной
температуре $u_{0}$ по всей длине. Найти момент времени $T$ , в который температура в середине 
стержня будет наибольшей.

\begin{displaymath}
	\left\{
		\begin{array}{l}
			c \rho \frac{\partial u}{\partial t}=\frac{\partial}{\partial x}(K(u(x)\frac{\partial u}{\partial x}), x \in (0, 1), t > 0 \\
			u(x, 0) =  \varphi (x)  \\
			u(0, t) =  \mu (t)  \\
			K(u) \frac{\partial u}{\partial x} \vert_{x = l} = W(t)
		\end{array} \right.
\end{displaymath}

\section{Данные варианта}
Коэффициент теплопроводности отпределяется по следующей формуле:
\begin{equation}
	K(u) = a + bu^{\sigma}
\end{equation}
,где $a = 5$, $b= 0.1$, $\sigma=2.5$.

На левом конце стержня поддерживется постоянная температура $u = u_{0}$. На правый конец стержня подаётся тепловой 
поток, который задаётся следующей формулой:
\begin{displaymath}
	W(t) = \left\{
		\begin{array}{ll}
			Q & ,0 \le t <  t_{0} \\
			0 & ,t \ge t_{0}
		\end{array} \right.
\end{displaymath}
, где $t_{0} = 0.5$, $Q = 10$.

Отсальные, необходимые для решения задачи, величины имеют следующий значения:
$c=2$, $\rho = 1$,  $l = 1$, , $u_{0} = 0.05$.

В начальный момент времени $t = 0$ стержень находится при фиксированной температуре $u_{0}$ по всей длинe,
поэтому получаем $u(0, t) = \mu (t) = u_{0}$. На левом конце стержня поддерживается постоянная температура 
$u_{0}$, поэтому получаем $u(x, 0) = \varphi (x) = u_{0}$.

\chapter{Решение задачи}
...

\end{document}




