\documentclass[a4paper,12pt]{report}

\usepackage[utf8x]{inputenc}
\usepackage[russian,english]{babel} 
\usepackage[T2A]{fontenc}

\usepackage{mathtext}
\usepackage{tabularx}
\usepackage{array}
\usepackage{amsmath}

%\usepackage{ucs}
\usepackage{geometry}
\geometry{left=2cm}
\geometry{right=1.5cm}
\geometry{top=1cm}
\geometry{bottom=2cm}

\author{Фроловский Алексей \\ ИУ7-17 \\ Вариант 13}
\title{Лабораторная работа №3 \\ по курсу << Mетоды вычислений >>  \\ на тему: \\  
<< Уравнение Пуассона >> }
\date{4 июня 2013г.}

% Убираем слово Chapter# перед названием главы
\makeatletter
\def\@makechapterhead#1{%
	\vspace*{50\p@}%
	{
		\parindent \z@ \raggedright \normalfont
		%\ifnum \c@secnumdepth >\m@ne
		%    \huge\bfseries \@chapapp\space \thechapter
		%    \par\nobreak
		%    \vskip 20\p@
		%\fi
		\interlinepenalty\@M
		\Huge \bfseries #1\par\nobreak
		\vskip 40\p@
	}
}
\makeatother

\begin{document}

\maketitle

\chapter{Постановка задачи}
\section{Формулировка задачи}
Найти решение краевой задачи для уравнения Пуассона в прямоугольнике $0 \le x \le a$,
$0 \le y \le b$ в следующей формулировке:

\begin{displaymath}
	\left\{
		\begin{array}{l}
			\Delta u + f = 0 \\
			u\vert_{x=0} =  \varphi_{0}(y)  \\
			u\vert_{y=0} =  \psi_{0}(x)  \\
			u\vert_{x=a} =  \varphi_{a}(y) \\
			u\vert_{y=b} =  \psi_{b}(x)
		\end{array} \right.
\end{displaymath}

\section{Данные варианта}
Дано $a = 1$, $b = 1$, $\varphi_{0}(y) = 3sin\pi y$, $\varphi_{a}(y) = 2y$, $\psi_{0}(x) = 3x(1-x)$, 
$\psi_{b}(x) = 2x$ , $f(x, y) = 4x^{2}(y-y^{2})$.

\chapter{Решение задачи}
\section{Разностная схема}
Данная задача может быть решена с использованием схемы "крест". Тогда разностная схема имеет
следующий вид:

\begin{displaymath}
	\left\{
		\begin{array}{l}
			\dfrac{u_{i-1, j} - 2u_{ij} + u_{i+1, j}}{h_{x}^{2}} + \dfrac{u_{i, j-1} - 2u_{ij} + u_{i, j+1}}{h_{y}^{2}} + f_{ij} = 0 \\
			u_{0j} = \mu_{j} \\
			u_{Nj} = \nu_{j}  \\
			u_{i0} = \varphi_{i} \\
			u_{iN} = \psi_{i}
		\end{array} \right.
\end{displaymath}
где $i = \overline{0, N}$, $j = \overline{0, N}$.

Сетка при этом определяется шагами по осям $x$ и $t$:
\begin{equation}
	h_{x} = a / N_{x}, h_{y} = b / N_{y}, x = ih_{x}, y = jh_{y},
\end{equation}
где $N_{x}$ - количество шагов по оси $x$, $N_{y}$ - количество шагов по оси $y$, $a$ и $b$ - размеры приямойгольника.

Для решения данной разностной схемы можно использовать метод релаксации. Тогда значени функции на следующем
шаге будет определяться по следующим формулам:
\begin{equation}
	u_{ij}^{k+1/2} = \dfrac{h_{y}^{2}}{2(h_{x}^{2} + h_{y}^{2})} (u_{i-1,j}^{k+1} + u_{i+1, j}^{k}) +
		\dfrac{h_{x}^{2}}{2(h_{x}^{2} + h_{y}^{2})} (u_{i, j-1}^{k+1} + u_{j, j+1}^{k}) +
		\dfrac{h_{x}^{2}h_{y}^{2}}{2(h_{x}^{2} + h_{y}^{2})} f_{ij},
	u_{ij}^{k+1} = \omega u_{ij}^{k+1/2} + (1-\omega)u_{ij}^{k}.
\end{equation}

Значение параметра $\omega$ является постоянным коэффициентом, вычисляемым по формуле:
\begin{equation}
	\omega = \dfrac{2}{1 + sin\dfrac{\pi}{N}},
\end{equation}
где $N$ - количество шагов.

Количество итераций для метода релаксации оценивается по формуле:
\begin{equation}
	k_{рел} = \dfrac{K}{2\pi}N,
\end{equation}
где $K=ln(\varepsilon / ||x^{0}||)$, $x^{0}$ - начальное приближение.

В качестве условия останова используется следующее неравенство:
\begin{equation}
	||u^{k+1} - u^{k}|| < (2-\omega)\varepsilon.
\end{equation}

Выбранная разностная схема является абсолютно устойчивой. Порядок апроксимации имеет вид $O(h_{x}^{2} + h_{y}^{2})$.

\end{document}




